\Introduction

%КОММЕНТ ДО: где, для чего, кем используется система АЕМ. Какие есть проблемы с ней!
Adobe Experience Manager — система управления контентом, осуществляющая хранение, обработку и доставку различных видов контента, предназначенная для крупных компаний, имеющих потребности в управление быстро меняющимся контентом. Созданный контент публикуется на отдельных серверах, оптимизированных для быстрого и надежного чтения хранимых ресурсов. В компаниях использующих систему, может быть развернуто множество AEM сред, где каждая среда будет представлять из себя отдельный проект со своими уникальными модулями и контентом, над которыми ведется активная разработка и внедряются новые модули со своими конфигурациями. Хотя в AEM и имеются механизмы для отслеживания состояния системы, её модулей и компонентов, они проверяют лишь стандартные параметры системы, и не имеют необходимых конфигураций для настройки проверки новых, разработанных модулей и конфигураций.

Целью данной работы является разработка пакета управления мониторингом состояния системы Adobde Experince Manager и интеграция с программой мониторинга Nagios. Для достижения поставленной цели необходимо решить следующие задачи:

\begin{itemize}
\item проанализировать систему Adobe AEM и механизмы мониторинга имеющиеся в системе. Сделать выводы о необходимости разработки данного функционала.
\item Разработать пакет с проверками системы в соответствие с требованиями указанными заказчиком.
\item Реализовать интеграцию проверок с системой Nagios.
\item проверить работоспособность разработанного пакета.
\end{itemize}

В первой главе произведен обзор предметной области, сформулированы требования к разрабатываемому пакету и приведены проектные решения. Вторая глава описывает процесс разработки и тестирования пакета, процессы интеграции с системой Nagios и содержит пользовательскую документацию.