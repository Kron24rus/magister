\chapter{Анализ предметной области}
\label{cha:analysis}
%
% % В начале раздела  можно напомнить его цель
%

%\textbf{//КОММЕНТАРИЙ:
%Формулировать задачу обязательно до обзора системы? Если знать архитектуру системы, тогда понятно как такие требования получаются...}

\section{Формулировка задачи}
В данной главе будет рассмотрена система AEM и сформулированы формальные требования на основе первоначальных требований и возможностей системы.

Adobe Experience Manager — система управления контентом, осуществляющая хранение, обработку и доставку различных видов контента в масштабах предприятия. Система предназначена для крупных компаний, имеющих потребности в управление быстро меняющимся контентом, это могут быть как и внутренние корпоративные порталы так и порталы для внешних пользователей. Созданный контент публикуется на отдельных серверах, оптимизированных для быстрого и надежного чтения хранимых ресурсов.

Основные пользователи системы:
\begin{enumerate}
\item Авторы контента или контент-менеджеры - отвечают за наполнения сайтов контентом.
\item Администратор - отвечает за развитие, поддержку работоспособности и безопасности системы.
\item Разработчик - Разработчики работающие с системой занимаются разработкой новых компонентов, шаблонов, и функционала системы.
\item Внешние пользователи - имеют доступ к контенту созданному авторами контента и расположенному на экземплярах публикации.
\end{enumerate}

В компаниях использующих AEM может быть развернуто множество так называемых Лэндскейпов - Наборов элементов аппаратного обеспечения, программного обеспечения и средств, расположенных в определенной конфигурации. Лэндскейпы включают в себя различные наборы серверов в том числе с развернутыми на них AEM средами содержащими экземпляры авторов, публикации и диспетчеры (Приложение A рис.~\ref{fig:complexDeploy}). В крупных компаниях может быть развернуто множество таких сред представляющих из себя самостоятельные порталы или сайты со своим контентом. Такие порталы или сайты могут содержать скрытый контент, доступный только авторизованным внешним пользователям. Как правило, в таких компаниях внешние пользователи имеют один аккаунт для всех систем этой компании т.е применяется технология единого входа.

Несмотря на то что система AEM имеет богатый набор возможностей для авторов контента и администраторов системы, она не имеет встроенных функций для работы с внешними пользователями системы, а именно отсутствует управление авторизацией и контроль доступа для внешних пользователей системы.

Исходя из вышеописанных пунктов возникла необходимость доработки системы с целью реализации взаимодействия с развернутой в компании технологией единого входа и управления авторизацией внешних пользователей на сайтах и порталах компании использующих AEM. Исходные данные и требования полученные от компании: 
\begin{itemize} 
\item В компании развернут поставщик учетных записей, работающий по протоколу SAML.
\item В компании имеется три портала работающих на основе системы AEM.
\item Разрабатываемое решение должно легко внедрятся во все существующие и будущие порталы на базе AEM.
\item Разрабатываемое решение должно поддерживать технологии единого входа и  выхода.
\item Должна быть возможность использовать различные SAMl привязки, которые должны быть настраиваемыми в каждом приложение.
\item Данные пользователей не должны хранится в хранилище данных системы AEM. Должны хранится только данные сессии, которая имеет короткое время жизни.
\end{itemize}
% Обратите внимание, что включается не ../dia/..., а inc/dia/...
% В Makefile есть соответствующее правило для inc/dia/*.pdf, которое
% берет исходные файлы из ../dia в этом случае.

%\begin{figure}
%  \centering
%  \includegraphics[width=\textwidth]{inc/dia/rpz-idef0}
%  \caption{Рисунок}
%  \label{fig:fig01}
%\end{figure}
%
%В \cite{Pup09} указано, что...
%
%Кстати, про картинки. Во-первых, для фигур следует использовать \texttt{[ht]}. Если и после этого картинки вставляются <<не по ГОСТ>>, т.е. слишком далеко от места ссылки,~--- значит у вас в РПЗ \textbf{слишком мало текста}! Хотя и ужасный параметр \texttt{!ht} у окружения \texttt{figure} тоже никто не отменял, только при его использовании документ получается страшный, как в ворде, поэтому просьба так не делать по возможности.

\section{Обзор системы и существующих компонентов}

AEM – система построенная с использованием технологии Java. В основе системы лежит фреймворк OSGI. Фреймворк имеет свой контекст, в который и устанавливаются все модули, они взаимодействуют между собой по средствам сервисов зарегистрированных в регистре сервисов рис.~\ref{fig:servicePattern}. Под динамической системой понимается возможность устанавливать, удалять и обновлять модули без перезапуска системы.

\begin{figure}[h]
  \centering
  \includegraphics[width=\textwidth]{inc/svg/servicePattern}
  \caption{Диаграмма архитектуры OSGI}
  \label{fig:servicePattern}
\end{figure}

\subsection{Архитектура AEM}
Основу AEM составляют набор модулей, которые можно условно выделить в 4 компоненты:
\begin{enumerate}
\item Java content repository – один из типов объектной базы данных, созданных для хранения и извлечения иерархических данных. Данные в JCR представляют из собой дерево, состоящее из узлов с ассоциированными с ними свойствами. Эти свойства и являются хранимыми данными, и могут хранить строки, числа или основные примитивные типы данных, а так-же двоичные данные, изображения и.т.д. 
\item Apache Sling – веб-фреймворк построенный по архитектуре REST, отвечающий за доставку контента в контент-ориентированных приложениях с использованием JCR.
\item AEM модули – набор бандлов реализованных компанией Adobe с использованием вышеупомянутых технологий.
\item Пользовательские модули – модули, разрабатываемые разработчиками, и расширяющие функционал системы.
\end{enumerate}

\begin{figure}[h]
  \centering
  \includegraphics[width=\textwidth]{inc/dia/osgi}
  \caption{Архитектура AEM}
  \label{fig:fig02}
\end{figure}

%\textbf{//Комментарий: Если дальше буду говорить про какие-то фишки системы которые проверяю, т.е РЕПЛИКАЦИЯ, Режимы запуска, Жизненный цикл компонентов, то их тоже нужно тут указать?}
%Выделить какой-то абзац под описание ключевых фишек(компонентов) AEM
\subsection{Обзор SAML модулей системы}
В системе имеется SAML модуль - "Adobe Granite - SAML 2.0 Authentication Handler" \cite{web:aemSaml}, предназначенный для интеграции с поставщиком учетных записей. Данный модуль имеет конфигурацию представленную на рис.~\ref{fig:defaultHandlerConfig1} и рис.~\ref{fig:defaultHandlerConfig2}. Рассмотрим подробнее следующие параметры конфигурации:
\begin{itemize}
\item path - путь до контента по которому будет вызван данный обработчик.
\item idpUrl - URL поставщика учетных записей который обрабатывает запросы на авторизацию.
\item idpCertAlias - имя сертификата IdP.
\item serviceProviderEntityId - имя поставщика сервиса сохраненное в IdP.
\item keyStorePassword - пароль от хранилища сертификатов.
\item defaultRedirectUrl - URL на который пользователь будет перенаправлен после успешной авторизации.
\item useEncryption - шифровать сообщения с помощью сертификата из хранилища.
\item createUser - создавать пользователя внутри системы AEM если он не был создан ранее.
\item defaultGroups - группа в которую будет добавлен пользователь при создании.
\item synchronizeAttributes - синхронизировать атрибуты из ответа SAML с атрибутами созданного пользователя.
\item handleLogout - обрабатывать запросы на выход из приложения.
\item logoutUrl - URL выхода, по которому будет вызван данный обработчик.
\end{itemize}

Данный обработчик либо создает пользователя в системе при первой авторизации если выбран параметр "createUser" или требует уже созданного пользователя. Также требуется назначить данных пользователей в соответствующие группы, а также страницам должны быть назначены эти группы. 

Вывод: Несмотря на то что данный подход реализует технологию единого входа он использует хранилище на экземплярах публикации для хранения пользователей, что требует синхронизации между всеми экземплярами публикации и разработки дополнительного функционала для связи атрибутов пользователя с IdP с атрибутами  хранящимся в системе. Необходимость хранить данные пользователей в системе противоречит требованиям и может быть использовано только для экземпляров автора. Также конфигурация реализует только один тип SAML привязки - "HTTP POST binding". Существующий модуль не может удовлетворить поставленные требования, в связи с чем было принято решение разработать новый модуль который будет выступать в роли полноценного поставщика сервиса.

\section{Формальные требования}
Данный раздел уточняет и формирует требования к разрабатываемому модулю.

\subsection{Функциональные требования}
Осуществляет проверку наличия в системе всех сконфигурированных агентов репликации, их состояние, размер их очереди, и для активных агентов, с не переполненной очередью, выполняет тестовую репликацию. Наличие в системе активных агентов которые не проверяются и не игнорируются приведет к провалу проверки.

Требуемые параметры конфигурации:
\begin{itemize}
\item Имена агентов репликации наличие и состояние которых необходимо проверить.
\item Имена агентов репликации которые игнорируются во время проверки.
\item Максимально разрешенный размер очереди для агентов.
\end{itemize}
%END REPLICATION AGENTS

%APPLICATION BUNDLE CHECK
\subsection{Нефункциональные требования}
Осуществляет проверку наличия в системе всех перечисленных в конфигурации модулей, проверяет что они активны, и выполняет проверку активности и статуса всех сервис-компонент входящих в этот модуль.

Требуемые параметры конфигурации:
\begin{itemize}
\item Список полных символьных имен модулей, статус которых необходимо проверить.
\item Список имен компонентов, которые должны быть отключены.
\end{itemize}
%END APPLICATION BUNDLE CHECK

\section{Заключение}
В данной главе был произведен обзор системы и рассмотрены сценарии развертывания при которых возникает необходимость разработки пакета, расширяющего возможности мониторинга состояния системы. Встроенные механизмы мониторинга не имеют необходимых настроек чтобы покрыть требования заказчика, но предоставляют возможность расширения функционала мониторинга системы. Было принято решение разработать пакет расширяющий стандартный функционал мониторинга системы и реализовать интеграцию с системой мониторинга Nagios.


%%% Local Variables:
%%% mode: latex
%%% TeX-master: "rpz"
%%% End:
