\Defines % Необходимые определения. Вряд ли понадобться
\begin{description}
\item[Распределённый] Слово, которое нельзя употреблять. Но надо протестировать длинные строки в глоссарии.
\item[JCR] Java Content Repository - java API to store data.
\item[OSGI] Open gateway initiative
\item[Replication] Process of copying content from Author instance to Publish instance
\item[Авторский сервер (Author instance)] Сервер, предназначенный для создания контента и управления им. Данный сервер предназначен для работы авторов(контент-менеджеров), которые использую удобный интуитивно понятный интерфейс занимаются наполнением контента сайта. 
\item[Публичный сервер (Publish instance)] Сервер делает контент, создаваемый авторами, доступным для целевой аудитории (пользователей).
\item[Репликация] Механизм в AEM используемый для публикации (копирования) контента с Авторского сервера на Публичный сервер.
\item[Bundle] модули в контексте OSGI спецификации. Представляет из себя jar файл с дополнительной мета информацией.
\end{description}

%%% Local Variables:
%%% mode: latex
%%% TeX-master: "rpz"
%%% End:
