\chapter{Анализ предметной области}
\label{cha:analysis}
%
% % В начале раздела  можно напомнить его цель
%

%\textbf{//КОММЕНТАРИЙ:
%Формулировать задачу обязательно до обзора системы? Если знать архитектуру системы, тогда понятно как такие требования получаются...}

\section{Формулировка задачи}
Adobe Experience Manager — система управления контентом, осуществляющая хранение, обработку и доставку различных видов контента в масштабах предприятия. Система предназначена для крупных компаний, имеющих потребности в управление быстро меняющимся контентом, это могут быть как и внутренние корпоративные порталы так и порталы для внешних пользователей. Созданный контент публикуется на отдельных серверах, оптимизированных для быстрого и надежного чтения хранимых ресурсов.

Основные пользователи системы:
\begin{enumerate}
\item Авторы контента или контент-менеджеры - отвечают за наполнения сайтов контентом. В системе организована возможность удобного управления контентом портала, с использованием графического интуитивно понятного интерфейса, без перезагрузок и остановки работы портала, так как предполагается постоянное активное использование ресурсов пользователями. Авторам не требуются навыки программирования или разработки, и система предоставляет авторам следующие функции:
\begin{enumerate}
\item Возможность загружать в систему цифровые ресурсы, такие как картинки, видео, тексты и.др, 
\item Создавать страницы из шаблонов и редактировать существующие, наполнять страницы AEM компонентами и цифровыми ресурсами. 
\item Активировать созданный контент - т.е активировать процесс копирования ресурсов с сервера разработки на сервер публикации. 
\item Удалять не актуальный контент, в том числе и страницы или AEM компоненты и цифровые ресурсы с конкретных страниц. 
\item Копировать контент чтобы ускорить создание похожих ресурсов.
\end{enumerate}

\item Администратор - отвечает за развитие, поддержку работоспособности и безопасности системы. Для администратора предусмотрен графический веб-интерфейс отображающий информацию о состояние системы, компонентах и позволяющий производить настройку системы. Для работы администратора в системе предусмотрен следующий функционал:
\begin{enumerate}
\item Возможность загружать пакеты созданные разработчиками, или пакеты обновлений поставляемые разработчиками системы и устанавливать их. Пакеты могут содержать новые компоненты, шаблоны, цифровые ресурсы, файлы конфигураций, контент, модули системы. 
\item Возможность создавать пакеты с контентом - позволяет скачивать пакеты и переносить созданный контент на другие сервера с целью переиспользования уже созданного контента. 
\item Добавлять в систему новые модули, а так-же активировать, останавливать или удалять существующие. 
\item Останавливать, активировать и конфигурировать сервис-компоненты системы. 
\item Добавлять или удалять пользователей, группы пользователей, устанавливать права доступа. 
\item Выполнять тестирование системы по заданным стандартным параметрам, проводить диагностику отслеживать логи и состояние системы используя встроенные механизмы с заданными параметрами.
\end{enumerate}
\item Разработчик - Разработчики работающие с системой занимаются разработкой новых компонентов, шаблонов, и функционала системы. Для них также предусмотрена графическая веб-среда разработки. Архитектора позволяет разработчикам встраивать в систему новый функционал без необходимости останавливать работу системы и не нарушая работу других компонентов.
\end{enumerate}

Функционал предоставляемый системой полностью удовлетворяет потребности авторов контента а архитектура системы позволяет разработчикам легко разрабатывать и новый функционал для системы.

В компаниях использующих AEM может быть развернуто множество так называемых Лэндскейпов - Наборов элементов аппаратного обеспечения, программного обеспечения и средств, расположенных в определенной конфигурации. Лэндскейпы включают в себя различные наборы серверов в том числе с развернутыми на них AEM средами содержащими экземпляры авторов, публикации и диспетчеры (Приложение A рис.~\ref{fig:complexDeploy}). Каждая AEM среда представляет из себя отдельный проект со своими уникальными модулями и контентом, над которыми ведется активная разработка и внедряются новые модули со своими конфигурациями. Хотя в AEM и имеются механизмы для отслеживания состояния системы и её модулей и компонентов, они проверяют лишь основные параметры системы, и не имеют необходимых конфигураций для более тонкой настройки проверки. 

Вывод:
Исходя из вышеописанных пунктов возникла необходимость разработки модуля, расширяющего возможности мониторинга состояния системы, с возможностью задавать настройки проверок. Требуется отслеживать следующие показатели и информировать администратора в случае сбоев на любом из серверов:
\begin{itemize} 
\item Необходимо проверять механизмы копирования контента с экземпляров авторов на экземпляры публикации.
\item Необходимо проверять работоспособность модулей и их сервис-компонентов имеющих важное значение для работоспособности системы, и/или определенного функционала системы.
\item Проверять что режимы работы системы соответствует требуемым. 
\item Проверять доступность удаленных ресурсов от которых может зависеть работоспособность модулей и компонентов системы.
\item Предусмотреть возможность интеграции результатов проверок с программой мониторинга Nagios \cite{web:nagiosDocs}, для автоматического отслеживания их статуса и оповещения администратора в случае неисправностей.
\end{itemize}
% Обратите внимание, что включается не ../dia/..., а inc/dia/...
% В Makefile есть соответствующее правило для inc/dia/*.pdf, которое
% берет исходные файлы из ../dia в этом случае.

%\begin{figure}
%  \centering
%  \includegraphics[width=\textwidth]{inc/dia/rpz-idef0}
%  \caption{Рисунок}
%  \label{fig:fig01}
%\end{figure}
%
%В \cite{Pup09} указано, что...
%
%Кстати, про картинки. Во-первых, для фигур следует использовать \texttt{[ht]}. Если и после этого картинки вставляются <<не по ГОСТ>>, т.е. слишком далеко от места ссылки,~--- значит у вас в РПЗ \textbf{слишком мало текста}! Хотя и ужасный параметр \texttt{!ht} у окружения \texttt{figure} тоже никто не отменял, только при его использовании документ получается страшный, как в ворде, поэтому просьба так не делать по возможности.

\section{Обзор системы и используемых компонентов}

AEM – система построенная с использованием технологии Java. В основе системы лежит фреймворк OSGI.

\subsection{Архитектура OSGI}
OSGI – фреймворк для построения модульной, динамической системы в котором модуль называется бандл. Фреймворк имеет свой контекст, в который и устанавливаются все модули, они взаимодействуют между собой по средствам сервисов зарегистрированных в регистре сервисов рис.~\ref{fig:servicePattern}. Под динамической системой понимается возможность устанавливать, удалять и обновлять модули без перезапуска системы.

\begin{figure}[h]
  \centering
  \includegraphics[width=\textwidth]{inc/svg/servicePattern}
  \caption{Диаграмма архитектуры OSGI}
  \label{fig:servicePattern}
\end{figure}

OSGI спецификация концептуально разделяется на 3 уровня \cite{osgiInAction}:
\begin{enumerate}
\item Модульный уровень – уровень модулей, которые в терминах спецификации называются бандл (далее по тексту - модуль). Они представляют из себя JAR-архив с специальной мета-информацией. Модули содержат в себе java-классы и ресурсы. Модули могут реализовывать функционал системы, а так-же предоставлять сервисы для использования другими модулями. Дополнительная информация указывается в файле META-INF/MANIFEST.MF может иметь различные заголовки, ключевыми являются:
\begin{itemize}
\item Bundle-Name - имя модуля.
\item Bundle-SymbolicName - символьное имя, которое однозначно идентифицирует модуль.
\item Bundle-Version - указывает версию модуля.
\item Import-Package - заголовок указывает внешние зависимости модуля (импортируемые Java-пакеты из других модулей). Могут быть указаны конкретные версии или диапазоны версий для каждой зависимости.
\item Export-Package - заголовок указывает java-пакеты, которые видны вне модуля. Если пакет не объявлен в этом заголовке, он будет виден только внутри модуля.
\item Bundle-Activator - имя класса в котором заданы действия выполняющиеся при запуске модуля.
\item Service-Component - заголовок содержащий список XML-файлов описывающих сервис-компоненты модуля.
\end{itemize}

\item Уровень жизненного цикла – уровень, определяющий, и управляющий операциями жизненного цикла модулей, состоит из нескольких состояний. Жизненным циклом модуля управляет OSGI контейнер (рис.~\ref{fig:bundleLifeCycle}).

Этапы жизненного цикла модулей в OSGI:
\begin{itemize}
\item INSTALLED - модуль успешно установлен в систему.
\item RESOLVED - модулю доступны все необходимые зависимости и он готов к запуску.
\item STARTING - выполняются действия запуска модуля описанные в активаторе.
\item ACTIVE - модуль успешно запущен.
\item STOPPING - выполняются действия остановки модуля.
\item UNINSTALLED - модуль удален, означает завершение жизненного цикла модуля.
\end{itemize}

\begin{figure}
  \centering
  \includegraphics[width=\textwidth]{inc/svg/LifeCycle}
  \caption{Возможные состояния модуля в OSGI}
  \label{fig:bundleLifeCycle}
\end{figure}

\item Уровень сервисов - позволяет модулям взаимодействовать между собой с помощью сервисов. Во время запуска модули регистрируют объекты с реализацией заявленного интерфейса в регистре сервисов OSGI фреймворка (Приложение А рис.~\ref{fig:simpleServices}). Другие модули могут находить и использовать зарегистрированные сервисы. Другим подходом к регистрации сервисов являются декларативные сервисы - Суть декларативных сервисов заключается в том, что создается дескриптор сервиса: XML-файл, который описывает сервис. Затем данный файл регистрируется в MANIFEST файле модуля. Диаграмма жизненного цикла декларативных сервисов приведена в Приложении A рис.~\ref{fig:declareService}.

\end{enumerate}

\subsection{Архитектура AEM}
Основу AEM составляют набор модулей, которые можно условно выделить в 4 компоненты:
\begin{enumerate}
\item Java content repository – один из типов объектной базы данных, созданных для хранения и извлечения иерархических данных. Данные в JCR представляют из собой дерево, состоящее из узлов с ассоциированными с ними свойствами. Эти свойства и являются хранимыми данными, и могут хранить строки, числа или основные примитивные типы данных, а так-же двоичные данные, изображения и.т.д. 
\item Apache Sling – веб-фреймворк построенный по архитектуре REST, отвечающий за доставку контента в контент-ориентированных приложениях с использованием JCR.
\item AEM модули – набор бандлов реализованных компанией Adobe с использованием вышеупомянутых технологий.
\item Пользовательские модули – модули, разрабатываемые разработчиками, и расширяющие функционал системы.
\end{enumerate}

\begin{figure}[h]
  \centering
  \includegraphics[width=\textwidth]{inc/dia/osgi}
  \caption{Архитектура AEM}
  \label{fig:fig02}
\end{figure}

%\textbf{//Комментарий: Если дальше буду говорить про какие-то фишки системы которые проверяю, т.е РЕПЛИКАЦИЯ, Режимы запуска, Жизненный цикл компонентов, то их тоже нужно тут указать?}


%Выделить какой-то абзац под описание ключевых фишек(компонентов) AEM
AEM - серверное приложение, работающее на платформе Java. Оно доступно в виде исполняемого JAR файла (файл быстрого запуска). Установка приложения создаст экземпляр который может быть либо экземпляром автора либо публикации, что задается режимом запуска. В разработке используется несколько развернутых сред.
\begin{itemize}
\item Среда разработки - используется разработчиками для разработки новых модулей системы. Может состоять как из экземпляров авторов и публикации, так и только иметь экземпляр автора, в таком случае тестирование сразу проводится в тестовой среде.
\item Среда тестирования - тестовая среда в которой проверяется функционал разработанных модулей и дизайн шаблонов, компонентов и ресурсов.
\item Рабочая среда - среда состоящая из экземпляров авторов и публикации. Где авторы вносят актуальный контент в экземпляры публикации а пользователи имеют доступ к контенту через экземпляры публикации. Как правило в рабочей среде используется несколько раздельных серверов с развернутыми на них экземплярами публикации и их работой управляет диспетчер осуществляющий кэширование и распределение нагрузки между серверами.
\end{itemize}

Экземпляр - В терминологии AEM экземпляр является копией AEM, запущенной на сервере. Установленные экземпляры отличаются своими конфигурациями. Конфигурации задаются с помощью режимов работы, выделяют два основных типа экземпляров AEM:
\begin{itemize}
\item Авторский экземпляр - экземпляр как правило расположенный за внутренним фаерволом и предназначен для работы авторов контента. 
\item Экземпляр публикации - как правило располагаются на общедоступных серверах через которые посетители могут иметь доступ к сайту и взаимодействовать с ним.
\end{itemize}

Режим работы - Режимы работы позволяют настраивать экземпляр AEM для конкретной цели. Например, как экземпляры автора или публикации, тестирования, разработки или другие. Режимы работы позволяют задать параметры конфигурации системы и дополнительные модули которые необходимо установить при установки экземпляра. Режимы запуска делятся на 2 типа:
\begin{itemize}
\item Установочные - задаются в момент установки экземпляра, и их нельзя поменять при последующих перезапусках. К таким режимам относят:
\begin{enumerate}
\item author - определяет конфигурации для экземпляра автора (взаимоисключающий с publish)
\item publish - определяет конфигурации для экземпляра публикации (взаимоисключающий с author) 
\item samplecontent - установка пакетов с демонстрационных контентом (используется для тестирования и разработки, взаимоисключающий с nosamplecontent)
\item nosamplecontent - пустой экземпляр без дополнительного контента (используется для рабочих экземпляров, взаимоисключающий с samplecontent)
\end{enumerate}
\item Пользовательские - режимы запуска созданные пользователями. Они задают необходимые параметры конфигурации системы и модули которые необходимо установить. Эти режимы можно менять при каждом запуске экземпляра.
\end{itemize}

Репликация - процесс копирования контента с экземпляров автора на экземпляры публикации. Во время активации контента автором вручную, или по расписанию создается пакет, который потом копируется на экземпляр публикации и устанавливается. Для задания правил копирования, расписания и экземпляров на которые осуществляется копирование используются конфигурации которые называются агентами репликации.

\subsection{Обзор используемых модулей}
	В системе имеется механизм для отслеживания некоторых параметров и состояний - модули "Apache Sling Health Check Core" и "Apache Sling Health Check Web Console Plugin". С их помощью реализуется функционал выявления ошибок в системе и администратору предоставляется веб интерфейс управления функционалом выявления ошибок соответственно \cite{web:slingHealthCheck}.

Модуль "Apache Sling Health Check Web Console Plugin"
	С помощью этого модуля реализован интерфейс для администратора, в котором он может настроить параметры проверок, и осуществляется вывод результата проверок в веб-консоли. В веб интерфейсе можно задавать следующие параметры:
\begin{itemize}
\item Health Check tags - Список тэгов по которым будет отобраны проверки для выполнения.
\item Combine tags with logical 'OR' - Выбирать проверки с любым тэгом из списка тэгов.
\item Show DEBUG logs - Показывать дебаг логи.
\item Show failed checks only - Отображать только проверки с ошибками.
\item Override global timeout - Переопределить глобальное максимальное время выполнения проверки.
\end{itemize}

Модуль "Apache Sling Health Check Core"
	С помощью данного модуля в системе реализован функционал осуществляющий выполнение сервисов, позволяющих администратору определить наличие ошибок в системе, а в некоторых случаях и определить их причины. Модуль регистрирует в системе набор настроенных проверок. Для каждой проверки создается JMX MBean предоставляющий доступ к логу выполнения и результату. Также этот модуль предоставляет интерфейс позволяющий расширять функционал системы пользовательскими проверками. 
	 
	В результате анализа модуля "Apache Sling Health Check Core" были выявлены 3 проверки наиболее соответствующие заданным требованиям.
	
	Обзор имеющихся проверок приведен в таблице 1.1
		
\begin{table}[ht]
  \caption{Сравнение SAML \cite{web:wikiSaml} фреймворков}
  \begin{tabular}{|p{3cm}|p{47mm}|p{30mm}|p{35mm}|}
  \hline
  Название      
  & Поддержка 
  & Количество запросов на stackoverflow 
  & Примеры \\
  \hline
  OpenSAML 3  
  & Поддерживается, последняя версия: март 2017
  & 399
  & Есть примеры, книги \\
  \hline
  OneLogin       			   
  & Поддерживается, последняя версия: ноябрь 2018
  & 324
  & Есть примеры \\
  \hline
  Spring Security SAML                
  & Поддерживается, последняя версия: март 2019
  & боле 500
  & Есть примеры \\
  \hline
  Pac4j       			   
  & Поддерживается, последняя версия: февраль 2019
  & 16
  & Есть примеры \\
  \hline
  \end{tabular}
  \label{tab:tabular}
\end{table}	
	
	Обнаруженные проверки не могут удовлетворить потребности в полном объеме.
\begin{enumerate}
\item Replication and Transport Users - Не имеет конфигурации, а следовательно не выполняет требуемых проверок.
\item Active Bundles - Наиболее подходящая проверка, но выполняет избыточную проверку, необходимо проверять только заданные модули, проверка всех модулей может привести к ложным срабатываниям и не осуществляет проверку статуса компонент.
\item Replication Queue - Осуществляет проверку очереди, но не размер очереди, а количество попыток первого элемента, так-же нет конфигурации для проверки активных и неактивных агентов репликации.
\end{enumerate}

Вывод: Стандартные модули системы не позволяют реализовать требуемый функционал в полном объеме. Система предоставляет интерфейс с помощью которого можно расширить функционал проверок системы.

\section{Формальные требования}


%ПРОВЕРКА АГЕНТОВ РЕПЛИКАЦИИ
\subsection{Проверка агентов репликации}
Осуществляет проверку наличия в системе всех сконфигурированных агентов репликации, их состояние, размер их очереди, и для активных агентов, с не переполненной очередью, выполняет тестовую репликацию. Наличие в системе активных агентов которые не проверяются и не игнорируются приведет к провалу проверки.

Требуемые параметры конфигурации:
\begin{itemize}
\item Имена агентов репликации наличие и состояние которых необходимо проверить.
\item Имена агентов репликации которые игнорируются во время проверки.
\item Максимально разрешенный размер очереди для агентов.
\end{itemize}
%END REPLICATION AGENTS

%APPLICATION BUNDLE CHECK
\subsection{Проверка модулей конкретных приложений}
Осуществляет проверку наличия в системе всех перечисленных в конфигурации модулей, проверяет что они активны, и выполняет проверку активности и статуса всех сервис-компонент входящих в этот модуль.

Требуемые параметры конфигурации:
\begin{itemize}
\item Список полных символьных имен модулей, статус которых необходимо проверить.
\item Список имен компонентов, которые должны быть отключены.
\end{itemize}
%END APPLICATION BUNDLE CHECK

%PING HEALTH CHECK
\subsection{Проверка доступности удаленных ресурсов}
Осуществляет проверку доступности удаленных ресурсов заданных в конфигурации с помощью URL.

Требуемые параметры конфигурации:
\begin{itemize}
\item Список HTTP URL систем доступность которых требуется проверить.
\end{itemize}
%PING HEALTH CHECK END

%RUN MODES HEALTH CHECK
\subsection{Проверка режимов работы}
Осуществляет проверку соответствия режимов работы системы и режимов из конфигурации проверки. Только полное соответствие допустимо.

Требуемые параметры конфигурации:
\begin{itemize}
\item Список режимов работы которым должен соответствовать экземпляр
\end{itemize}
%END RUN MODES HEALTH CHECK

%NAGIOS INTEGRATION
\subsection{Интеграция с Nagios}
Предусмотреть возможность интеграции проверок с программой Nagios: Программа должна периодически опрашивать проверки и получать их статус и лог выполнения проверок, статус сервиса в Nagios должен соответствовать статусу проверки, в случае сбоя администратору должно отправляться оповещение по электронной почте.
%END NAGIOS INTEGRATION

\section{Заключение}
В данной главе был произведен обзор системы и рассмотрены сценарии развертывания при которых возникает необходимость разработки пакета, расширяющего возможности мониторинга состояния системы. Встроенные механизмы мониторинга не имеют необходимых настроек чтобы покрыть требования заказчика, но предоставляют возможность расширения функционала мониторинга системы. Было принято решение разработать пакет расширяющий стандартный функционал мониторинга системы и реализовать интеграцию с системой мониторинга Nagios.


%%% Local Variables:
%%% mode: latex
%%% TeX-master: "rpz"
%%% End:
