\Abbreviations %% Список обозначений и сокращений в тексте
\begin{description}
\item[JCR (Java content repository)] Тип объектной базы данных, используется в системе для хранения, поиска и извлечения иерархических данных.
\item[OSGI] Cпецификация для построения модульных систем для платформы Java \cite{web:osgiSite}.
%\item[Авторский сервер (Author instance)] Сервер, предназначенный для создания контента и управления им. Данный сервер предназначен для работы авторов(контент-менеджеров), которые использую удобный интуитивно понятный интерфейс занимаются наполнением контента сайта.
\item[Экземпляр автора (Author instance)] Среда, предназначенная для создания контента и управления им. Основные пользователи авторы(контент-менеджеры).
\item[Экземпляр публикации (Publish instance)] Среда, предоставляющая пользователям доступ к контенту созданному авторами.
%\item[Публичный сервер (Publish instance)] Сервер делает контент, создаваемый авторами, доступным для целевой аудитории (пользователей).
\item[SSO (Single Sign-On)] Технология единого входа пользователей, благодаря которой владея одной лишь учетной записью пользователь может посещать множество различных ресурсов.
\item[SAML (security assertion markup language)] Язык разметки, основанный на языке XML с помощью которого реализуется SSO. Является открытым стандартом обмена данными аутентификации и авторизации между участниками, в частности, между поставщиком учётных записей и поставщиком услуг.
\item[IdP (identity provider)] Поставщик учетных записей – центральный сервер который хранит аккаунты пользователей.
\item[SP (service provider)] Поставщик услуг – сервис на стороне приложения который обращается к IdP для авторизации пользователя.
\item[Бандл (bundle)] модули в терминах OSGI спецификации. Представляет из себя JAR архив с дополнительной мета информацией.
\item[Лэндскейп (landscape)] Набор элементов аппаратного обеспечения, программного обеспечения и средств, расположенных в определенной конфигурации.
\item[Диспетчер (dispatcher)] Это средство кэширования и / или балансировки нагрузки в Adobe Experience Manager.
\item[Режимы работы (run modes)] Позволяют настраивать экземпляры AEM для конкретных целей и задавать базовую конфигурацию.
\item[AEM Компонент] Настраиваемые компоненты, используемые авторами для наполнения страниц контентом.
\item[Сервис компонент (компонент)] Является основной составляющей декларативных сервисов OSGI спецификации.
%Суть DS в том, что создается дескриптор сервиса: XML-файл, который описывает сервис. Затем данный файл регистрируется в манифесте бандла. Соответственно, OSGi-среда после ресолвинга зависимостей данного бандла автоматически стартует описанные сервисы и переводит бандл в состояние ACTIVE. Естественно, что активатор бандла при этом выполняется.
\end{description}

Определения связанные с архитектурой AEM более подробно рассмотрены в параграфе 3 главы 1.

%%% Local Variables:
%%% mode: latex
%%% TeX-master: "rpz"
%%% End:
