\Introduction

%КОММЕНТ ДО: где, для чего, кем используется система АЕМ. Какие есть проблемы с ней!
Adobe Experience Manager — система управления контентом, осуществляющая хранение, обработку и доставку различных видов контента, предназначенная для крупных компаний, имеющих потребности в управление быстро меняющимся контентом. Созданный контент публикуется на отдельных серверах, оптимизированных для быстрого и надежного чтения хранимых ресурсов. В компаниях использующих систему, может быть развернуто множество AEM сред, где каждая среда будет представлять из себя отдельный проект со своими уникальными модулями и контентом доступным конечным пользователям системы. 

Экземпляр автора системы имеет различные роли пользователей рассчитанные на контент-менеджеров и администраторов системы. Аккаунты пользователей хранятся внутри системе в JCR хранилище. Если в среде компании имеется поставщик учетных записей и имеется множество систем на базе AEM, содержащих контент доступный только авторизованным пользователям, то встроенная система ролей и пользователей не может быть использована. В таком случае экземпляры публикации должны выступать в роли поставщика услуг и взаимодействовать с поставщиком учетных записей для реализации технологии единого входа.

Целью данной работы является разработка модуля управления авторизацией системы AEM с использованием стандарта SAML. Для достижения поставленной цели необходимо решить следующие задачи:

\begin{itemize}
\item Изучить механизмы доступные в системе AEM для реализации SSO и SLO. Сделать выводы о необходимости разработки нового функционала.
\item Изучить фреймворки реализующие стандарт SAML и их применимость в среде AEM, выбрать наиболее подходящий фремворк.
\item Разработать модуль реализующий SSO и SLO с использованием стандарта SAML и выбранного ранее фреймворка.
\item Проверить работоспособность разработанного модуля в среде компании на всех порталах и с развернутым поставщиком учетных записей.
\end{itemize}

В первой главе произведен обзор предметной области, сформулированы требования к разрабатываемому модулю, выбран фреймворк который будет использован при разработке модуля. Во второй главе приведены проектные решения с учетом выбранного в первой главе фреймворка. Треья глава описывает процесс разработки, тестирования модуля а также пользователькую документацию.